\documentclass{article}
\usepackage{biblatex}
\addbibresource{refs.bib}
\usepackage{comment}
\usepackage[utf8]{inputenc}
\author{BS}
\usepackage{amsfonts} 
\usepackage{amsmath}
\usepackage{amsthm}
\usepackage{amssymb}
\newcommand{\C}{\mathbb{C}}
\newcommand{\R}{\mathbb{R}}
\newcommand{\Z}{\mathbb{Z}}
\renewcommand{\S}{\mathbb{S}}
\begin{document}
\section*{List of topics for qualification exam (Bruno)}
\subsection*{Basic Topics from standard courses:}
\subsubsection*{Riemannian Geometry \cite{docarmo}, \cite{besse2007einstein}, \cite{jost2008riemannian}}
\begin{itemize}
    \item Basics: Geodesics, Levi-Civita connection, Curvature
    \item Geodesic Completness and Hopf-Rinow Theorem
    \item Berger's list of Special Holonomy
\end{itemize}
\subsubsection*{Symplectic Geometry \cite{Arnold}, \cite{guillemin1990symplectic}, \cite{da2001lectures}}
\begin{itemize}
    \item Moser's trick on compact manifolds and applications.
%    \begin{itemize}
%        \item Moser theorem on family of cohomology classes
%        \item Darboux-Weinstein Theorem on local form around closed subset
%        \item Weinstein's Lagrangian neighbourhood theorem
%    \end{itemize}
    \item Arnold-Liouville Theorem, Examples of Integrable Systems
    \item Weinstein-Marsden Theorem on Symplectic reduction
\end{itemize}
\subsubsection*{Complex Geometry \cite{huybrechts2005complex}, \cite{griffiths2014principles}, \cite{Voisin_2002}}
\begin{itemize}
	\item Divisors and Line bundles, rational maps to projective space
	\item Harmonic Forms and Hodge Decomposition of Cohomology for Kahler
        \item Hodge Index Theorem and Lefschetz hyperplane theorem
	\item Atiyah classes of holomorphic vector bundles and holomorphic connections
\end{itemize}

\subsection*{Topics outside standard courses:}
\subsubsection*{Topics on Quantization}
\begin{itemize}
    \item Geometric/Holomorphic quantization of Kähler manifolds \cite{Schlichenmaier_2010}
%    \begin{itemize}
%        \item Berezin-Toeplitz operators on quantizable compact kahler manifolds and classical limit ($m>>1$) \cite{Bordemann_1994}, \cite{Ma_2008}
%    \end{itemize}
    \item Deformation Quantization
    \begin{itemize}
        \item Kontsevich's theorem about DQ of Poisson manifolds. \cite{Kontsevich_2003}
        \item Example via Berezin-Toeplitz operators \cite{schlichenmaier1999deformationquantizationcompactkaehler}.
%        \item Kontsevich's formula via Perturbative poisson $\sigma$-model.    
\end{itemize}
\end{itemize}
\subsubsection*{Topics on Modern Mathematical Physics \cite{QFT-Strings,Mirror}}
\begin{itemize}
    \item Notions of Quantum Field Theory
    \item Supersymmetric $\sigma$-models and special holonomies
    \item Aspects of Mirror Symmetry
\end{itemize}
\clearpage


\subsubsection*{Jacobian Conjecture \cite{BCW}}
\begin{itemize}
    \item Results contained in Bass-Connel-Wright (82). %Enough to prove injective (many proofs). Enough to prove $F$ is proper. Equivalent over any field of characteristic 0.
    \item Botanical Series for formal inverse of any invertible formal map $F$ (in particular any holomorphic function $F$ with $F(0)=0$ and $DF(0)=\mathrm{Id}$).
    \item Formulation via Wick's sum ($0d$ QFT) \cite{Abdesselam}
\end{itemize}

\subsubsection*{Rozansky--Witten Theory and Chern--Simons Theory}
\begin{itemize}
    \item The weight systems: linear algebra data and graphs. IHX relation. \cite{sawon2004rozanskywitteninvariantshyperkahlermanifolds}
    \item Application of Rozansky--Witten weight system with Chern numbers in compact Hyperkahler manifolds \cite{hitchin1999curvaturecharacteristicnumbershyperkahler}.
    \item The Quantum Field Theories: \cite{qftjones}, \cite{rozansky1997hyperkahlergeometryinvariantsthreemanifolds}
    \begin{itemize}
        \item Chern--Simons Theory as a Gauge Theory on a trivial $G$-bundle over compact $3$-manifold $M$. Quantization \cite{Axelrod:1989xt} on cylinder $M=\Sigma\times \R$:
$$
\mbox{ Geometric Quantization of }\mathrm{Hom}(\pi_1(\Sigma),G)/G.
$$
    \item Rozansky--Witten Theory as 3-dimensional topological sigma-model with a hyperkähler manifold as target space.
    \end{itemize}
\end{itemize}

\printbibliography
\end{document}
